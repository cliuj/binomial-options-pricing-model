\documentclass[12pt, letterpaper]{article}\usepackage{float}
\usepackage{amsmath}
\usepackage{hyperref}
\usepackage{indentfirst}

\usepackage[superscript, nospace, nobreak, sort]{cite}
\renewcommand\citeform[1]{[#1]}
\renewcommand\citepunct{}


\usepackage{enumitem}[labelindent=5cm]

\usepackage[bottom]{footmisc}

\usepackage{tikz}
\usepackage{caption}
\usetikzlibrary{matrix, positioning}
\tikzset{bullet/.style={circle,fill,inner sep=2pt}}

\usepackage{graphicx}
\graphicspath{ {./images/} }

\usepackage{geometry}
\geometry{margin=1in}

\begin{document}
\setlength\parindent{1cm}

\title{Binomial Options Pricing Model (CRR Model)\\
\large MATH 3332 - \textit{Probability and Inference}}
\author{Charlie Liu}
\date{\today}
\maketitle

\section*{Introduction}

In finance and economics, the asset market is a blanket term that covers all the various underlying markets that involve the exchange of financial assets.
The stock market is one famous example of an underlying market where buyers and sellers can exchange assets in the form of stocks (or shares) of various companies.
Other notable markets include the bond market where fixed-income assets like corporate bonds are exchanged, and the derivatives market where contracts between two consenting parties are exchanged.
The accessibility of these markets depend on the features provided by the service or brokerage used.

\medskip

Since assets exchanged on the market are based on a seller's chosen price, one must account for the valuation of the asset to avoid over paying.
The valuation of an asset is based on the future expected cash flow\cite{damodaran} and is especially important when exchanging assets because pricing is typically determined by sellers and influenced by buyers.
It is important to note here that because asset valuation is based on future expected cash flow, any calculated valuation such as intrinsic\footnote{True value based on quantitative analysis} valuation is considered theoretical.
This is why quantitative analysis is heavily used in asset valuation as the various mathematical models are used to estimate the valution of an asset.

\medskip

For financial derivatives, valuation is difficult to calculate because they are essentially contracts that \textit{derive} their value from the performance of the underlying entity\cite{derivativefinancewikipedia}, thus various factors in addition to the factors affecting the underlying stock value can alter the valuation of the option contract. 

\medskip

In this paper, we will focus on a specific financial derivative called the options contract with stocks as its derivation.
An options contract gives the owner the right to ``exercise" the contract which means the owner can either buy or sell the ${x}$ number\footnote{The number of underlying stock in a single option is usually 100 stocks.} of underlying stock at a fixed price called the stike price once the underlying stock's price passes the strike price\footnote{In the case of ``puts" (right to sell), the stock's price must be lower than the strike price. In the case of ``calls" (right to buy), the stock's price must be higher than the strike price.\label{option rules}}.
They can be viewed as reservations to buy or sell ${x}$ number of stock.

\medskip

Once the strike price is reached or passed, the options contract is considered ``in-the-money"\footnote{In-the-money depends on the type of the option\footref{option rules}.} and can be reasonably exercised.
Anything else is considered ``out-of-the-money".
In order to profit from exercising the option, the amount generated from exercising at the strike price must cover the cost\footnote{The options contract's premium.} of the option itself.
All option contracts come with an expiration time\footnote{Also known as ``maturity"}.
If the option expires out-of-the-money, it is considered worthless\footnote{\$0} because the owner of the option can get a better deal at market price instead of the option's strike price.
An option contract to buy the underlying entity is called a call and an option contract to sell the underlying entity is called a put.
One may buy a call or put option on the market and then sell that same option on the market.

\medskip

The famous mathematical model called the Black-Scholes Model developed by Fischer Black, Myron Scholes, and Robert C. Merton was able to give a theoretical valuation of European-style options \cite{blackscholesmodelwikipedia}.
There is emphasis on ``European-style" options in this case because the Black-Scholes model (by itself) is only accurate on the European-style options.
European-style options only allow exercise on expiration, while American-style options allow the holder to exercise before expiration.
This early exercise of the American-style options makes it difficult to find the optimal time to exercise the option for maximum gains \cite{blackscholesmodelwikipedia} as the Black-Scholes model cannot\footnote{Because it is essentially a black-box.} account for the various outcomes before expiration. 

\medskip

The binomial options pricing model overcame this limitation for the American-style options \cite{bopmwikipedia}.
Its main feature of generating different outcomes in the form of ``nodes" allows for better insight into the optimal stop.
It was originally proposed by William Sharpe in 1978, but was formalized in 1979 by John Carrington Cox, Stephen Ross, and Mark Edward Rubenstein \cite{bopmwikipedia}.

\medskip

The name ``binomial options pricing model" is a generic name that encompasses all forms\footnote{Different versions i.e using different formulas} of the binomial options pricing models.
Cox, Ross, and Rubenstein's version of the binomial options pricing model became commonly known as the ``CRR model" \cite{thebinomialmodelcornell}.

\medskip

In this paper, ``binomial options pricing model" will be synonymous to the CRR model and for the sake of keeping consistency and simplicity, all formulas and calculations demonstrated here will be based on the CRR model.

\pagebreak

\section*{Methods}

The binomial options pricing model aims to find the valuation of the chosen option at discrete times based on various inputs given.
It achieves this by generating a binomial decision tree (lattice) where each node represents the valuation of the option at a certain point in time.
This binomial lattice is composed of Bernoulli processes because at each (non end/start) node there are two directional outcomes of the stock: up or down.
This means that the probabilities for up and down movement are ${p}$ and ${1-p}$ respectively where ${0 \leq p \leq 1}$. 
Figure \ref{fig:genericlattice} gives a generalized 2-period example of a binomial options lattice this:
\begin{figure}[H]
  \centering
  \begin{tikzpicture}[>=stealth, sloped, baseline]
    \matrix (tree) [%
      matrix of nodes,
      minimum size=1cm,
      column sep=2cm,
      row sep=1.2cm, nodes={text width=5em}
      	]
      {
  		  t=0 & t=1 & t=2 \\
  
            &  &  {} \\
  
        	&  {}  &\\
  
        {}  &  &  {}\\
  
        	&  {}  &\\
  
            &  &  {} \\
      };
      \node[bullet, left=0mm of tree-2-3.west,
        label={[label distance=1mm]90:${u^2S_0}$},
        label={[label distance=2mm]0:${XV_{t,n}}$}
        ](b-2-3){};

      \node[bullet, left=0mm of tree-3-2.west,
        label={[label distance=1mm]90:${uS_0}$},
        label={[label distance=2mm]0:${XV_{t,n}}$},
        label={[label distance=2mm]270:${BV_{t,n}}$}
        ](b-3-2){};

      \node[bullet, right=7mm of tree-4-1.west,
        label={[label distance=1mm]90:${S_0}$},
        label={[label distance=2mm]0:${XV_{t,n}}$},
        label={[label distance=2mm]270:${BV_{t,n}}$}
        ](b-4-1){};

      \node[bullet, left=0mm of tree-4-3.west,
        label={[label distance=1mm]90:${udS_0}$},
        label={[label distance=2mm]0:${XV_{t,n}}$}
        ](b-4-3){};

      \node[bullet, left=0mm of tree-5-2.west,
        label={[label distance=1mm]90:${dS_0}$},
        label={[label distance=2mm]0:${XV_{t,n}}$},
        label={[label distance=2mm]270:${BV_{t,n}}$}
        ](b-5-2){};

      \node[bullet, left=0mm of tree-6-3.west,
        label={[label distance=1mm]90:${d^2S_0}$},
        label={[label distance=2mm]0:${XV_{t,n}}$}
        ](b-6-3){};

      \draw[->] (b-4-1) -- (b-3-2) node [midway,above] {$p$};
      \draw[->] (b-4-1) -- (b-5-2) node [midway,below] {$(1-p)$};
      
  		\draw[->] (b-3-2) -- (b-2-3) node [midway,above] {$p$};
      \draw[->] (b-3-2) -- (b-4-3) node [midway,below] {$(1-p)$};
  
  		\draw[->] (b-5-2) -- (b-4-3) node [midway,above] {$p$};
      \draw[->] (b-5-2) -- (b-6-3) node [midway,below] {$(1-p)$};
      \matrix [draw,below left] at (current bounding box.south) {
        \node [label=right:${S_0 =}$ \text{initial stock price}] {}; \\
        \node [label=right:${XV_{t, n} =}$ \text{exercise value at time period ${t}$}] {}; \\
        \node [label=right:${BV_{t, n} =}$ \text{binomial value at time period ${t}$}] {}; \\
      };
  \end{tikzpicture}
  \caption{A generic binomial options pricing lattice with 2 periods}
  \label{fig:genericlattice}
\end{figure}

\pagebreak
There are multiple ways to obtain the valuation of the option using this model.
The novel feature of this model is that option valuation can be achieved without the need of probabilities.
This same feature is what made Black-Scholes famous in that the true probabilities and expected returns were uneccessary to obtain the valuation of the given option contract\cite{brigidavideo}.
Two arbitrage-free pricing methods through ``delta-hedging" and replicating portfolio can achieve the same valuation and both do not need the probabilities at each outcome.
Both of these methods are outside the scope of this paper, and instead this paper will focus on the probability method used to obtain the value of an option.

\bigskip

The process to evaluate the given option is described as a three step process\cite{bopmwikipedia}:
\begin{enumerate}[leftmargin=2\parindent]
  \item Price Tree Generation
  \item Calculation of Exercise Values
  \item Backward Induction for Binomial Values
\end{enumerate}

\bigskip

To best describe this method, it's best to have an example.
For this example, we will calculated the valuation of an American-style call option with the following inputs:
\begin{align*}
  \setlength{\jot}{20pt}
  \text{Let } &\text{\textit{(S): Stock Price}} = 100 \\
  \text{Let } &\text{\textit{(X): Strike Price}} = 120 \\
  \text{Let } &\text{\textit{(r): Risk-Free Interest Rate}} = 0.09\% \\
  \text{Let } &\text{\textit{(u): Up Factor}} = 1.25 \\
  \text{Let } &\text{\textit{(d): Down Factor}} = 1/u = 1/1.25 = 0.80 \\
  \text{Let } &\text{\textit{(T): Time in Year(s)}}  = 1 \\
  \text{Let } &\text{\textit{(t): Time Periods}}  = 2 
\end{align*}

\bigskip

Explaination for the newly introduced variables:
\begin{itemize}[label={}]
  \item \textbf{Risk-Free Interest Rate} - Will be explained in the upcoming section
  \item \textbf{Up Factor} - The factor that the stock price moves up at for each period (t)
  \item \textbf{Down Factor} - The factor that the stock price moves down at for each period (t)
  \item \textbf{Time in Years} - The time until expiration in years\footnote{So if our duration is 0.5, then our entire model is lasting half a year}.
  \item \textbf{Time Periods} - The number of time periods (generations) in the binomial lattice
\end{itemize}

\pagebreak
\subsection*{1. Price Tree Generation}
\subsubsection*{Establishing Movement Factors} \label{establishing movement factors}
The initial step in generating the lattice is to obtain the up(${u}$) and down(${d}$) factors. From the original 1979 CRR paper, the up and down factors are calculated using\cite{bopmwikipedia, crrpaper}:
\begin{align*}
  u &= e^{\sigma\sqrt{x/y}} \\
  d &= e^{-\sigma\sqrt{x/y}} = \frac{1}{u}
\end{align*}
\begin{align*}
  \text{where}~\sigma &= \text{the volatility (standard deviation) of the stock price}, \\
  x &= \text{the duration of a time period ${(t)}$ out of ${T}$ years}, \\
  y &= \text{the duration until expiration ${(T)}$}
\end{align*}
\paragraph{Note:} The unit of time for both ${x}$ and ${y}$ must be the same. \\

\bigskip
The rate of increase in the movement factors: ${\sigma\sqrt{x/y}}$ and ${-\sigma\sqrt{x/y}}$ were \textit{probably} derived from the mean absolute deviation (MAD) formula since both fomulas bear a resemblence to the MAD's formula\cite{madwikipedia}:
\begin{equation*}
  MAD = \sigma\sqrt{2/\pi}
\end{equation*}

\noindent Which seems reasonable in this context as both ${u}$ and ${d}$ are essentially compounding the volatility (standard deviation) over a period of time from the initial price of a stock.

\bigskip

\subsubsection*{Lattice Structure}

One neat thing about these formulas is that the stock price at a certain time period ${n}$ can be calculated with:
\begin{gather*}
  S_n = S_0 \cdot u^{N_u-N_d} \\
  \begin{align*}
    \text{where}~N_{n \in \{u,d\}} &= \text{number of movements (${u}$, ${d}$) taken for a path}
  \end{align*} \\
\end{gather*}

\medskip

Nodes with paths ${u > d}$:
\begin{equation*}
  \setlength{\jot}{10pt}
  \begin{split}
    {S}_n
    & = {S}_0 \cdot {u}^{N_u-N_d} \\
    & = {S}_0 \cdot {u}^n
  \end{split}
\end{equation*}

\pagebreak

Nodes with paths ${u < d}$:
\begin{equation*}
  \setlength{\jot}{10pt}
  \begin{split}
    {S}_n
    & = {S}_0 \cdot {u}^{N_u-N_d} \\
    & = {S}_0 \cdot {u^{-n}} \\
    & = {S}_0 \cdot {1/u^n} \\
    & = {S}_0 \cdot ({1/u})^n \\
    & = {S}_0 \cdot {d^n}
  \end{split}
\end{equation*}

\bigskip

Nodes with paths ${u = d}$:
\begin{equation*}
  \setlength{\jot}{10pt}
  \begin{split}
    {S}_n
    & = {S}_0 \cdot {u}^{N_u-N_d} \\
    & = {S}_0 \cdot {u^0} \\
    & = {S}_0 \cdot 1 \\
    & = {S}_0
  \end{split}
\end{equation*}

\medskip

\noindent Hence, the lattice structure. These formulas are derived from the continous compouding formula\cite{continouscompoundingformulawikipedia, exponentcharacterizationswikipedia}:
\begin{equation*}
  P(t) = {P_0}e^{rt}
\end{equation*}
because the value at each consecutive node is essentially the compounded value of the initial stock price at the rate of the movement factor.


\subsubsection*{Risk-free interest rate}
Risk-free\footnote{Realistically, truly risk-free is rare or else it's technically arbitrage} interest rate is the value that represents the safest interest yield one would earn if they were to put the money in that asset.
This is is typically used as the expected minimum return for any asset. 
The most common risk-free interests used are the U.S Treasury bond yields as the bonds are considered the safest\footnote{The risk is so low that its negligible} assets to obtain interest.

\medskip

For simplicity, this paper will use 0.09\%\footnote{This was the yield listed on November 25, 2020} from the current 6-month\footnote{6-months because in our example, we have our expiration at 1 year and only 2 time periods i.e two 6-month periods} U.S Treasury bill as the risk-free interest rate for calculations.

\pagebreak
\subsubsection*{Generated Tree}
Given the variables for this example, the tree with stock price estimates is generated below:
\begin{figure}[H]
  \centering
  \begin{tikzpicture}[>=stealth, sloped, baseline]
    \matrix (tree) [%
      matrix of nodes,
      minimum size=1cm,
      column sep=2cm,
      row sep=1.2cm, nodes={text width=5em}
      	]
      {
  		  t=0 & t=1 & t=2 \\
  
  					&  &  {} \\
  
        	&  {}  &\\
  
        {}  &  &  {}\\
  
        	&  {}  &\\
  
  					&  &  {} \\
      };

      \node[bullet, left=0mm of tree-2-3.west,
        label={[label distance=1mm]90:${\$156.25}$},
        label={[label distance=2mm]0:${XV_{t,n}}$}
        ](b-2-3){};

      \node[bullet, left=0mm of tree-3-2.west,
        label={[label distance=1mm]90:${\$125}$},
        label={[label distance=2mm]0:${XV_{t,n}}$},
        label={[label distance=2mm]270:${BV_{t,n}}$}
        ](b-3-2){};

      \node[bullet, right=7mm of tree-4-1.west,
        label={[label distance=1mm]90:${\$100}$},
        label={[label distance=2mm]0:${XV_{t,n}}$},
        label={[label distance=2mm]270:${BV_{t,n}}$}
        ](b-4-1){};

      \node[bullet, left=0mm of tree-4-3.west,
        label={[label distance=1mm]90:${\$100}$},
        label={[label distance=2mm]0:${XV_{t,n}}$}
        ](b-4-3){};

      \node[bullet, left=0mm of tree-5-2.west,
        label={[label distance=1mm]90:${\$80}$},
        label={[label distance=2mm]0:${XV_{t,n}}$},
        label={[label distance=2mm]270:${BV_{t,n}}$}
        ](b-5-2){};

      \node[bullet, left=0mm of tree-6-3.west,
        label={[label distance=1mm]90:${\$64}$},
        label={[label distance=2mm]0:${XV_{t,n}}$}
        ](b-6-3){};

      \draw[->] (b-4-1) -- (b-3-2) node [midway,above] {$p$};
      \draw[->] (b-4-1) -- (b-5-2) node [midway,below] {$(1-p)$};
      
  		\draw[->] (b-3-2) -- (b-2-3) node [midway,above] {$p$};
      \draw[->] (b-3-2) -- (b-4-3) node [midway,below] {$(1-p)$};
  
  		\draw[->] (b-5-2) -- (b-4-3) node [midway,above] {$p$};
      \draw[->] (b-5-2) -- (b-6-3) node [midway,below] {$(1-p)$};
      \matrix [draw,below left] at (current bounding box.south) {
        \node [label=right:${XV_{t, n} =}$ \text{exercise value at time period ${t}$}] {}; \\
        \node [label=right:${BV_{t, n} =}$ \text{binomial value at time period ${t}$}] {}; \\
      };
  \end{tikzpicture}
  \caption{The generated binomial tree with stock prices for 2 periods}
  \label{fig:lattice_with_stock_prices}
\end{figure}


\pagebreak
\subsection*{2. Calculation of Exercise Values}
With the price tree generated in Figure~\ref{fig:lattice_with_stock_prices}, the option's exercise\footnote{This is also referred to as the intrinsic (face) value} value needs to be found for each node because American-style call options can be exercised at any time before expiration.
For a European-style call option, only the final nodes of the lattice will have the option value because European-style options can only be exercised at expiration. 
The exercise value is calculated \cite{bopmwikipedia} by:
\begin{align*}
  \text{\textit{Call Option (${XV_{t,n}}$)}} &= Max({(S_n - X)}, 0) \\
  \text{\textit{Put Option (${XV_{t,n}}$)}} &= Max(0, {(X - S_n)}) 
\end{align*}

\bigskip

For a call option, if the difference between the stock price and the strike price ${(S_n - X)}$ is negative, then the option is worthless\footnote{\$0} because it is cheaper for the option owner to purchase the shares at market price rather than the strike price.
The same is applied for the put option, except put options gives the user the right to sell the underlying stock at the certain strike price, therefore the setup for the difference is reversed ${(X - S_n)}$.
If the difference is negative, then option is also worthless because it is more profitable to sell the underlying stocks at market price rather than the strike price.

\pagebreak
Our updated tree with the exercise value of the options:
\begin{figure}[H]
  \centering
  \begin{tikzpicture}[>=stealth, sloped, baseline]
    \matrix (tree) [%
      matrix of nodes,
      minimum size=1cm,
      column sep=2cm,
      row sep=1.2cm, nodes={text width=5em}
      	]
      {
  		  t=0 & t=1 & t=2 \\
  
  					&  &  {} \\
  
        	&  {}  &\\
  
        {}  &  &  {}\\
  
        	&  {}  &\\
  
  					&  &  {} \\
      };

      \node[bullet, left=0mm of tree-2-3.west,
        label={[label distance=1mm]90:${\$156.25}$},
        label={[label distance=2mm]0:${\$36.25}$},
        label={[label distance=2mm]270:${BV_{t,n}}$}
        ](b-2-3){};

      \node[bullet, left=0mm of tree-3-2.west,
        label={[label distance=1mm]90:${\$125}$},
        label={[label distance=2mm]0:${\$5}$},
        label={[label distance=2mm]270:${BV_{t,n}}$}
        ](b-3-2){};

      \node[bullet, right=7mm of tree-4-1.west,
        label={[label distance=1mm]90:${\$100}$},
        label={[label distance=2mm]0:${\$0}$},
        label={[label distance=2mm]270:${BV_{t,n}}$}
        ](b-4-1){};

      \node[bullet, left=0mm of tree-4-3.west,
        label={[label distance=1mm]90:${\$100}$},
        label={[label distance=2mm]0:${\$0}$},
        label={[label distance=2mm]270:${BV_{t,n}}$}
        ](b-4-3){};

      \node[bullet, left=0mm of tree-5-2.west,
        label={[label distance=1mm]90:${\$80}$},
        label={[label distance=2mm]0:${\$0}$},
        label={[label distance=2mm]270:${BV_{t,n}}$}
        ](b-5-2){};

      \node[bullet, left=0mm of tree-6-3.west,
        label={[label distance=1mm]90:${\$64}$},
        label={[label distance=2mm]0:${\$0}$},
        label={[label distance=2mm]270:${BV_{t,n}}$}
        ](b-6-3){};

      \draw[->] (b-4-1) -- (b-3-2) node [midway,above] {$p$};
      \draw[->] (b-4-1) -- (b-5-2) node [midway,below] {$(1-p)$};
      
  		\draw[->] (b-3-2) -- (b-2-3) node [midway,above] {$p$};
      \draw[->] (b-3-2) -- (b-4-3) node [midway,below] {$(1-p)$};
  
  		\draw[->] (b-5-2) -- (b-4-3) node [midway,above] {$p$};
      \draw[->] (b-5-2) -- (b-6-3) node [midway,below] {$(1-p)$};
      \matrix [draw,below left] at (current bounding box.south) {
        \node [label=right:${BV_{t, n} =}$ \text{binomial value at time period ${t}$}] {}; \\
      };
  \end{tikzpicture}
  \caption{Updated tree with option value at the end of the 2 periods}
\end{figure}

\pagebreak
\subsection*{3. Backward Induction for Binomial Values}
With the exercise values calculated in the previous step, backward induction is used to obtain the option values at each node excluding the nodes at expiration.
The core concept of this step revolves around the Fundamental Theorem of Asset Pricing which implies that the asset value at time ${t}$ is the discounted value of its future value at expiration\footnote{Basically prices today is cheaper than what it will be at expiration.}\cite{blythevideo}\cite{ftapwikipedia}.
This discounted value in the context of the binomial options pricing model is called the ``binomial value" and is calculated from the ``risk neutral valuation".

\subsubsection*{Risk-neutral Valuation}
The risk neutral valuation is the valution of an option with risk neutrality assumption\cite{riskneutralvalutionwikipedia} which is based on the risk-neutral measure\footnote{Also known as the ``equivalent martingale measure"}\cite{riskneutralmeasurewikipedia}.
Jan Stuller and Kevin on: \href{https://quant.stackexchange.com/questions/55239/what-is-the-risk-neutral-measure}{\color{blue}{quant.stackexchange}} can explain these terms much better than I can.

\medskip

The most important take away from this, is the arbitrage-free\footnote{There are no arbitrage (ways to make profit with zero risk)} assumption and market completeness\footnote{In a complete market, any reasonable payoff can be replicated\cite{riskneutralmeasurestackexchange}}.
Both are tied to the existence of martingale\footnote{Future expected values are equal to the present expected value} measures stated by the fundamental theorems of asset pricing\cite{ftapwikipedia}, which implies what was said in the beginning: present value is the discounted future value.


\subsubsection*{Binomial Value}
Recall from earlier that the binomial value is essentially the discounted value of the option from its future value.
This is where the backwards induction comes in, as the value of the options (nodes) at time period ${t}$ are calculated based on the values obtained at time period ${t+1}$.

\bigskip

The binomial value formula\footnote{This was modified from Wikipedia to match this paper's variable usage} is:

\begin{gather*}
  BV_{t, n} = \frac{(p)V_{t+1,u} + (1-p)V_{t+1,d}}{e^{rt}} \\
  \\
  \begin{align*}
    \text{where}~V_{t,n} &= \text{the valuation of node ${n}$ where ${n \in \{u, d\}}$}, \\
    p &= \text{the probability of a up movement}, \\
    t &= \text{the time period}, \\
    e^{rt} &= \text{the compounded interest rate: } (1 + r)^t
  \end{align*} \\
\end{gather*}

\pagebreak

\noindent which in plain english is:
\begin{equation*}
  \text{\textit{Binomial Value at period ${t}$}} = \frac{\text{[${p}$ ${\cdot}$ \textit{Up Value} + (1 - ${p}$) \textit{Down Value}]}}{\text{\textit{Compounded risk-free interest at period ${t}$}}}
\end{equation*}

\bigskip

And this originated from the formula\footnote{This took me a long long time to break down and find out}\footnote{This also actually originated from Euler's Equation of Consumption\cite{riskneutralmeasurestackexchange}(I could not find the original formula) } for present value \cite{presentvaluewikipedia}:
\begin{equation*}
  P = \frac{F}{(1+r)^t}
\end{equation*}

\noindent Where ${F}$ represents the future value and in this context will be the expected value of returns of a node at period ${t+1}$:
\begin{equation*}
  F = (p)V_{t+1,u} + (1-p)V_{t+1,d}
\end{equation*}
and we know that expected value is discrete because we are calculating with 2 definite outcomes: up and down, thus we get\cite{hussainteaching}:
\begin{equation*}
  \setlength{\jot}{10pt}
  \begin{split}
    F = E[V] 
    & = \sum\limits_{n \in \{u,d\}} p_n V_{t,n} \\
    & = p_1 V_{t,u} + p_2 V_{t,d} \\
    & = (p) V_{t,u} + (1-p) V_{t,d} \\
  \end{split}
\end{equation*}

\subsubsection*{Probability}
Now that we're given the formula for the binomial value and broken down the reason for the formula, the last unknown to calculate is the probability ${p}$. This probability is calculated from the formula:
\begin{equation*}
  p = \frac{e^{rt} - d}{u - d}
\end{equation*}

\noindent which is essentially the formula for Andrew D. Roy's ratio \cite{riskneutralmeasurewikipedia, sharperatiowikipedia, royssafetyfirstcriterionwikipedia} to measure market price of risk:
\begin{gather*}
  \setlength{\jot}{10pt}
  \frac{\mu - m}{\sigma} \\
  \\
  \begin{align*}
  \text{where}~\mu &= \text{the gross expected return}, \\
  m &= \text{the minimum acceptable return (risk)}, \\
  \sigma &= \text{the standard deviation of returns}
  \end{align*} \\
\end{gather*}

\pagebreak
By solving for ${p}$ with our given example variables:
\begin{equation*}
  \setlength{\jot}{10pt}
  \begin{split}
    p
    & = \frac{e^{rt} - d}{u - d} \\
    & = \frac{(1+r)^t - d}{u - d} \\
    & = \frac{1.0009 - 0.80}{1.25 - 0.80} \\
    & = \frac{1.0009 - 0.80}{0.45} \\
    & = \frac{0.2009}{0.45} \\
    & = 0.6\overline{4}
  \end{split}
\end{equation*}

We can see that the final result fits the formula for market price of risk:
\begin{equation*}
  \frac{\mu - m}{\sigma} = \frac{1.0009 - 0.80}{0.45}
\end{equation*}

\noindent ${\mu}$ is ${1.0009}$ because the minimum expected return should be at least\footnote{Anything lower does not make sense, since you would take on more risk for a reward lower than the safest asset return.} the amount returned from the safest interest asset i.e treasury bonds. ${m}$ is ${0.80}$ and is based off the down factor because that is in the worst case, the amount of return\footnote{Returns can be negative.} one would get from the underlying purchase. ${\sigma}$ is ${0.45}$ and that is basically the standard deviation from the stock price at time ${t}$ to the up and down outcomes. Putting these together, the result is something that calculates the probability of getting a minimum acceptable return \cite{sfratioinvestopedia}.

\medskip

This probability effectively becomes the \textit{theoretical} probability of the option moving up and thus conversely by complement ${(1-p)}$ becomes the probability of the option moving down. With this in mind, the probabilities for our example are:
\begin{align*}
  p
  & \approx 0.45 \\
  (1-p)
  & \approx 0.55
\end{align*}

\pagebreak
\subsubsection*{Putting Everything Together}
With all the variables known, we can calculate the binomial values for all the nodes\footnote{From second to last period to ${0^{th}}$ period} except the expiration\footnote{Because the exercise value becomes the max value for the option at expiration.} nodes.
Since we are using an American-style option for this example, we must also take into consideration that we can exercise\footnote{For European-style options, the binomial value is applied to all nodes\cite{bopmwikipedia}} at any node before expiration.
This means that we need to find maximum value between the calculated Binomial Value and the Exercise Value denoted\cite{bopmwikipedia}:
\begin{equation*}
  V_{t,n} = Max({BV, XV})
\end{equation*}

\noindent which in plain English is:
\begin{equation*}
  \textrm{\textit{Option Value}} = Max(\text{\textit{Binomial Value}}, \text{\textit{Exercise Value}})
\end{equation*}

\bigskip

Now with our variables known and formulas established, we calculate the binomial value for each node (except the final nodes) and find the option's value at each node:
\begin{equation*}
  \setlength{\jot}{10pt}
  \begin{split}
    BV_{1, uS}
    & = Max(\frac{[(0.45)(36.25) + (0.55)(0)]}{(1+0.0009)^1}, 5) \\
    & \implies \approx\$16.30 \\
    \\
    BV_{1, dS}
    & = Max(\frac{[(0.45)(0) + (0.55)(0)]}{(1+0.0009)^1}, 0) \\
    & \implies \$0 \\
    \\
    BV_{0, S_0}
    & = Max(\frac{[(0.45)(16.30) + (0.55)(0)]}{(1+0.0009)^0}, 0) \\
    & \implies \approx\$7.33
  \end{split}
\end{equation*}

\pagebreak
Our final updated tree:
\begin{figure}[H]
  \centering
  \begin{tikzpicture}[>=stealth, sloped, baseline]
    \matrix (tree) [%
      matrix of nodes,
      minimum size=1cm,
      column sep=2cm,
      row sep=1.2cm, nodes={text width=5em}
      	]
      {
  		  t=0 & t=1 & t=2
        \\
  					&  &  ${}$
        \\
        	&  {}  &
        \\
        ${}$  &  &  ${}$
        \\
        	&  {}  &
        \\
  					&  &  ${}$
        \\
      };
      \node[bullet, left=0mm of tree-2-3.west,
        label={[label distance=1mm]90:${\$156.25}$},
        label={[label distance=2mm]0:${\$36.25}$}
        ](b-2-3){};

      \node[bullet, left=0mm of tree-3-2.west,
        label={[label distance=1mm]90:${\$125}$},
        label={[label distance=2mm]0:${\$5}$},
        label={[label distance=2mm]270:${\$16.30}$}
        ](b-3-2){};

      \node[bullet, right=10mm of tree-4-1.west,
        label={[label distance=1mm]90:${\$100}$},
        label={[label distance=2mm]0:${\$0}$},
        label={[label distance=2mm]270:${\$7.33}$}
        ](b-4-1){};

      \node[bullet, left=0mm of tree-4-3.west,
        label={[label distance=1mm]90:${\$100}$},
        label={[label distance=2mm]0:${\$0}$}
        ](b-4-3){};

      \node[bullet, left=0mm of tree-5-2.west,
        label={[label distance=1mm]90:${\$80}$},
        label={[label distance=2mm]0:${\$0}$},
        label={[label distance=2mm]270:${\$0}$}
        ](b-5-2){};

      \node[bullet, left=0mm of tree-6-3.west,
        label={[label distance=1mm]90:${\$64}$},
        label={[label distance=2mm]0:${\$0}$}
        ](b-6-3){};

      \draw[->] (b-4-1) -- (b-3-2) node [midway,above] {$0.45$};
      \draw[->] (b-4-1) -- (b-5-2) node [midway,below] {$0.55$};
      
  		\draw[->] (b-3-2) -- (b-2-3) node [midway,above] {$0.45$};
      \draw[->] (b-3-2) -- (b-4-3) node [midway,below] {$0.55$};
  
  		\draw[->] (b-5-2) -- (b-4-3) node [midway,above] {$0.45$};
      \draw[->] (b-5-2) -- (b-6-3) node [midway,below] {$0.55$};
  \end{tikzpicture}
  \caption{Updated tree with option value at the end of the 2 periods}
\end{figure}

\noindent We end up with the present day value of the option: ${\$7.33}$.

\subsection*{Summary}
In general, the binomial option's pricing model is a model in which all future returns are calculated and used to work backwards to obtain the values at each node.
The fundamental formula that this model relies is the present value formula because the calculation for the binomial value is essentially the calculation for the present value, and thus satisfies the implication from the Fundamental Theorem of Asset Pricing\cite{blythevideo}.

\pagebreak
\subsection*{Checking and Verifying the Work}
To verify that the example process and results are indeed correct, we will plug in the same information into Jan R\"{o}man's Binomial calculator using the CRR model\cite{romancalc}.
While the inputs are similar, R\"{o}man's Binomial calculator does not take the up ${(u)}$ and down ${(d)}$ factor as inputs.
Instead the up and down are calculated from the inputed time periods\footnote{R\"{o}man's calculator calls it ``steps"} (${t}$) and volatility (${\sigma}$).
Luckily, we can obtain these values from our given variables using the up ${(u)}$ formula introduced earlier in section \ref{establishing movement factors}{\textbf{Establishing Movement Factors}}:
\begin{gather*}
  u = e^{\sigma\sqrt{x/y}} \\
  \begin{align*}
    \text{where}~x &= \text{the duration of a time period ${(t)}$ out of ${T}$ years}, \\
    y &= \text{the duration until expiration ${(T)}$}
  \end{align*} \\
\end{gather*}

In our example, we are using \textit{days} as our unit of time because ${T=1}$ year and we have time period ${t=2}$, therefore ${y = 365}$ and ${x \approx 182}$. With this information and our up (${u}$) factor is ${1.25}$, we solve for the volatility${(\sigma)}$ of the underlying stock:
\begin{equation*}
  \setlength{\jot}{10pt}
  \begin{split}
    u
    & = e^{\sigma\sqrt{t/n}} \\
    1.25
    & = e^{\sigma\sqrt{\frac{182}{365}}} \\
    \implies \ln{(1.25)} &= \sigma\sqrt{\frac{182}{365}} \\
    \implies \sigma &= \frac{\ln{(1.25)}}{\sqrt{\frac{182}{365}}} \\
    & \approx 0.316
  \end{split}
\end{equation*}

Since the calculator requires that we input a percentage, we turn this into value into a percentage:
\begin{equation*}
    \sigma (\%) = 0.316 * 100 \implies 31.6\%
\end{equation*}

\pagebreak

\noindent Inputting these into the calculator, we get:
\begin{figure}[H]
  \includegraphics[scale=0.8]{Roman_binomial_calculator_inputs}
  \caption{R\"{o}man's Binomial Options Calculator Inputs}
\end{figure}

\begin{figure}[H]
  \includegraphics{Roman_binomial_calculator_stock_prices}
  \caption{R\"{o}man's Binomial Options Calculator Stock Prices Results}
\end{figure}

\begin{figure}[H]
  \includegraphics{Roman_binomial_calculator_option_values}
  \caption{R\"{o}man's Binomial Options Calculator Option Values Results}
\end{figure}

Comparing the results of Roman's calculator with the results of our example binomial tree, we can see that both stock and options values generated are very close.
The small differences of precision could be due to the rounded values.
Overall, the results are near identical and confirms that our calculation and proces was correct.

\pagebreak

\section*{Application}
For the application part of this paper, a binomial options pricing model program written in Python3 was made.
The program follows and uses the methods and formulas described in this paper and can calculate at a much larger number of steps.
The source code for this application can be found in the same repository that houses the source .tex file of this PDF, accessible here:
\href{https://github.com/cliuj/binomial-options-pricing-model}{\color{blue}{https://github.com/cliuj/binomial-options-pricing-model}}

To verify, we check if the output of the program is similar to the output of the earlier example above and R\"{o}man's calculator results:

\begin{figure}[H]
  \includegraphics[scale=0.66]{bopm_output_screenshot.png}
  \caption{Results of the Binomial Options Pricing Calculator Application}
\end{figure}

\pagebreak
Unfortunately, formatting is off if we use a bigger number of time periods:
\begin{figure}[H]
  \includegraphics[scale=0.55]{bopm_output_screenshot_25steps.png}
  \caption{Results of the Binomial Options Pricing Calculator Application with 25 Periods}
\end{figure}

\pagebreak
Luckily, it still outputs the present day value of the option:
\begin{figure}[H]
  \includegraphics[scale=0.55]{bopm_output_screenshot_25steps_PV.png}
  \caption{Results of the Binomial Options Pricing Calculator Application with 25 Periods with Present Value}
\end{figure}



\pagebreak
% It's too much of a pain to learn about BibTex, so I'm writing the old-fashion way.
% It's also too much of a pain to sort these...
\begin{thebibliography}{9}
  \bibitem{derivativefinancewikipedia}
    "Derivative (finance)". \\
    \href{https://en.wikipedia.org/wiki/Derivative\_(finance)}{https://en.wikipedia.org/wiki/Derivative\_(finance)}

  \bibitem{blackscholesmodelwikipedia}
    "Black-Scholes Model". \\
    \href{https://en.wikipedia.org/wiki/Black-Scholes\_model}{https://en.wikipedia.org/wiki/Black-Scholes\_model}

  \bibitem{bopmwikipedia}
    "Binomial options pricing model". \\
    \href{https://en.wikipedia.org/wiki/Binomial\_options\_pricing\_model}{https://en.wikipedia.org/wiki/Binomial\_options\_pricing\_model}

  \bibitem{volatilityfinancewikipedia}
    "Volatility (finance)". \\
    \href{https://en.wikipedia.org/wiki/Volatility\_(finance)}{https://en.wikipedia.org/wiki/Volatility\_(finance)}

  \bibitem{madwikipedia}
    "Average absolute deviation" \\
    \href{https://en.wikipedia.org/wiki/Average\_absolute\_deviation}{https://en.wikipedia.org/wiki/Average\_absolute\_deviation}

  \bibitem{continouscompoundingformulawikipedia}
    "Compound interest". \\
    \href{https://en.wikipedia.org/wiki/Compound\_interest}{https://en.wikipedia.org/wiki/Compound\_interest}

  \bibitem{exponentcharacterizationswikipedia}
    "Characterizations of the exponential function". \\
    \href{https://en.wikipedia.org/wiki/Characterizations\_of\_the\_exponential\_function}{https://en.wikipedia.org/wiki/Characterizations\_of\_the\_exponential\_function}

  \bibitem{riskneutralmeasurewikipedia}
    "Risk-neutral measure" \\
    \href{https://en.wikipedia.org/wiki/Risk-neutral\_measure}{https://en.wikipedia.org/wiki/Risk-neutral\_measure}

  \bibitem{riskneutralvalutionwikipedia}
    "Rational pricing". \\
    \textit{Risk neutral valuation}. \\
    \href{https://en.wikipedia.org/wiki/Rational\_pricing\#Risk\_neutral\_valuation}{https://en.wikipedia.org/wiki/Rational\_pricing\#Risk\_neutral\_valuation}

  \bibitem{ftapwikipedia}
    "Fundamental theorem of asset pricing". \\
    \href{https://en.wikipedia.org/wiki/Fundamental\_theorem\_of\_asset\_pricing}{https://en.wikipedia.org/wiki/Fundamental\_theorem\_of\_asset\_pricing}

  \bibitem{presentvaluewikipedia}
    "Present value"
    \href{https://en.wikipedia.org/wiki/Present\_value}{https://en.wikipedia.org/wiki/Present\_value}

  \bibitem{sharperatiowikipedia}
    "Sharpe Ratio". \\
    \href{https://en.wikipedia.org/wiki/Sharpe\_ratio}{https://en.wikipedia.org/wiki/Sharpe\_ratio}

  \bibitem{royssafetyfirstcriterionwikipedia}
    "Roy's safety-first criterion"
    \href{https://en.wikipedia.org/wiki/Roy\%27s\_safety-first\_criterion}{https://en.wikipedia.org/wiki/Roy\%27s\_safety-first\_criterion}

  \bibitem{blythevideo}
    Stephen Blythe.
    "20. Option Price and Probability Duality". \\
    \href{https://www.youtube.com/watch?v=eG\_aRPy1KVE}{https://www.youtube.com/watch?v=eG\_aRPy1KVE}, \\
    22:16-26:00
  
  \bibitem{damodaran}
    Aswath Damodaran.
    \textit{Investment Valuation: Tools and Techniques for Determining the Value of Any Asset, Third Edition}.
    [Chapter 5: Option Pricing Theory and Models], 87-110,
    ISBN: \textit{9781118011522},
    (2012)

  \bibitem{thebinomialmodelcornell}
    "The Binomial Model". \\
    \href{https://pi.math.cornell.edu/\~mec/Summer2008/spulido/Binomial.html}{https://pi.math.cornell.edu/\~mec/Summer2008/spulido/Binomial.html}

  \bibitem{brigidavideo}
    Matt Brigida.
    "FIN 376: Binomial Option Pricing and Delta Hedging".
    3:40-4:25. \\
    \href{https://www.youtube.com/watch?v=PZrmOh2nZus}{https://www.youtube.com/watch?v=PZrmOh2nZus}

  \bibitem{crrpaper}
    John Carrington Cox, Stephen Ross, and Mark Edward Rubenstein.
    "Option pricing: A simplified approach".
    (1979)
    \href{http://static.stevereads.com/papers\_to\_read/option\_pricing\_a\_simplified\_approach.pdf}{http://static.stevereads.com/papers\_to\_read/option\_pricing\_a\_simplified\_approach.pdf}\footnote{This is the original paper, but not the official hosting site.}

  \bibitem{tbondyield}
    "Daily Treasury Yield Curve Rates".
    \href{https://www.treasury.gov/resource-center/data-chart-center/interest-rates/Pages/TextView.aspx?data=yield}{https://www.treasury.gov/resource-center/data-chart-center/interest-rates/Pages/TextView.aspx?data=yield}

  \bibitem{riskneutralmeasurestackexchange}
    "Jan Stuller and Kevin". \\
    \href{https://quant.stackexchange.com/questions/55239/what-is-the-risk-neutral-measure}{https://quant.stackexchange.com/questions/55239/what-is-the-risk-neutral-measure}

  \bibitem{hussainteaching}
    Riaz Hussain.
    "3. BASICS OF PORTFOLIO THEORY". \\
    \href{https://www.scranton.edu/faculty/hussain/teaching/fin586\_/GPT103.pdf}{https://www.scranton.edu/faculty/hussain/teaching/fin586\_/GPT103.pdf},
    17-19

  \bibitem{sfratioinvestopedia}
    Will Kenton.
    "Roy's Safety-First Criterion (SFRatio) Definition". \\
    \href{https://www.investopedia.com/terms/r/roys-safety-first-criterion.asp}{https://www.investopedia.com/terms/r/roys-safety-first-criterion.asp}

  \bibitem{romancalc}
    Jan R\"{o}man.
    "Binomial-tree Option Calculator" \\
    \href{http://janroman.dhis.org/calc/Binomial2.php}{http://janroman.dhis.org/calc/Binomial2.php
}


\end{thebibliography}

\end{document}
